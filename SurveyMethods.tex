%%%%%%%%%%%%%%%%%%%%%%%%%%%%%%%%%%%%%%%%%
% Thin Sectioned Essay
% LaTeX Template
% Version 1.0 (3/8/13)
%
% This template has been downloaded from:
% http://www.LaTeXTemplates.com
%
% Original Author:
% Nicolas Diaz (nsdiaz@uc.cl) with extensive modifications by:
% Vel (vel@latextemplates.com)
%
% License:
% CC BY-NC-SA 3.0 (http://creativecommons.org/licenses/by-nc-sa/3.0/)
%
%%%%%%%%%%%%%%%%%%%%%%%%%%%%%%%%%%%%%%%%%

%----------------------------------------------------------------------------------------
%	PACKAGES AND OTHER DOCUMENT CONFIGURATIONS
%----------------------------------------------------------------------------------------

\documentclass[12pt]{article} % Font size (can be 10pt, 11pt or 12pt) and paper size (remove a4paper for US letter paper)

\usepackage[protrusion=true,expansion=true]{microtype} % Better typography
\usepackage{graphicx} % Required for including pictures
\usepackage{wrapfig} % Allows in-line images

\usepackage{mathpazo} % Use the Palatino font
\usepackage[T1]{fontenc} % Required for accented characters
\linespread{1.05} % Change line spacing here, Palatino benefits from a slight increase by default

\makeatletter
\renewcommand\@biblabel[1]{\textbf{#1.}} % Change the square brackets for each bibliography item from '[1]' to '1.'
\renewcommand{\@listI}{\itemsep=0pt} % Reduce the space between items in the itemize and enumerate
                                % environments and the bibliography

\usepackage{longtable}
\usepackage{caption}
\usepackage{url}
\usepackage{float}
\usepackage{times}
\usepackage{multirow}
\usepackage{listings}
\usepackage{times}
\usepackage{paralist}
\usepackage{epsfig}
\usepackage{subfigure}
\usepackage{longtable}
%\usepackage[hypertex]{hyperref}
\usepackage{subfigure}
\usepackage{color}
\usepackage{ifpdf}
\usepackage{float}
\usepackage{texdraw}
\usepackage{epsf}
\usepackage{array}
\usepackage{cite}
\usepackage{enumitem}
\usepackage{verbatim}
\usepackage{setspace}
\sloppy
\usepackage{geometry}
\usepackage{amsmath}

\renewcommand{\maketitle}{ % Customize the title - do not edit title and author name here, see the TITLE block below
\begin{flushright} % Right align
{\LARGE\@title} % Increase the font size of the title

\vspace{50pt} % Some vertical space between the title and author name

{\large\@author} % Author name
\\\@date % Date

\vspace{40pt} % Some vertical space between the author block and abstract
\end{flushright}
}

%----------------------------------------------------------------------------------------
%	TITLE
%----------------------------------------------------------------------------------------

\title{\textbf{Methodology and Design of UITS Satisfaction Survey and Analysis}\\ % Title
Confirming the Delight of Users Across the State} % Subtitle

\author{\textsc{Scott Michael, Takyua Noguchi, and Yan Zhou} % Author
\\{\textit{Research Analytics,\\
                University Information Technology Services,\\ 
                Indiana University, Bloomington, IN, 47408}}} % Institution

\date{\today} % Date

%----------------------------------------------------------------------------------------

\begin{document}

\maketitle % Print the title section

%----------------------------------------------------------------------------------------
%	ABSTRACT AND KEYWORDS
%----------------------------------------------------------------------------------------

%\renewcommand{\abstractname}{Summary} % Uncomment to change the name of the abstract to something else

\begin{abstract}
Every year the Research Analytics group constructs, administers, and analyzes survey data collected from the
main Indiana University campuses (at Bloomington and Indianapolis) and some subset of the regional
campuses. This report summarizes the methodology an procedures employed to conduct and analyze the
survey. Generally speaking, the survey questions are collected from the various UITS divisions and
stakeholders and compiled into a single survey instrument. This instrument is then distributed by the IU
Center for Survey Research (CSR) and results are collected. The CSR then provides the Research Analytics team
with the raw survey data which are analyzed to produce satisfaction measures for the services that the survey
is based around.
\end{abstract}

\hspace*{3,6mm}\textit{Keywords:} Data Analysis, Survey Methodology % Keywords

\vspace{30pt} % Some vertical space between the abstract and first section

%----------------------------------------------------------------------------------------
%	ESSAY BODY
%----------------------------------------------------------------------------------------

\section*{Introduction}
The UITS survey is administerd yearly to assess the s


%------------------------------------------------

\section*{Survey Design and Methodology}

\subsection*{Contacting Campuses for Survey Participants}



%------------------------------------------------

\section*{Analysis of Survey Data}

{\bf (This section will begin by describing the data format and delivery by CSR. The framework for the
  analysis follows)}

The following analysis is based primarily on \cite{Levy}. Let $N$ be the total population number of all
possible survey participants (i.e. the total count of all students, faculty and staff) and $n$ be the number
of participants who responded to a given survey item. For a given {\it strata} or sub-population $h$, $N_h$
represents the total number of possible survey participants in that sub-population (e.g. $N_{\mathrm{fac}}$ is
the total number of faculty members) and $n_h$ is the number of participants from that sub-population who
responded to a given survey item. 

Following from box 5.2 from \cite{Levy}, to evaluate the sample mean of a particular response to a Likert style
question $\bar{x}$ we first compute the sample mean for each of the strata for which the question was
administered. For individual responses $x_i$ the sample mean for a strata $\bar{x}_h$ is
\begin{equation}
\bar{x}_h=\frac{\sum\limits_{i=1}^{n_h}x_{h,i}}{n_h}
\end{equation} 
we can then weight the strata sample means according to the population numbers, where the weight for a
particular strata is defined as
\begin{equation}
W_h =\frac{N_h}{N}
\end{equation}
and
\begin{equation}
\bar{x}=\frac{\sum\limits_{h=1}^L N_h \bar{x}_h}{N}=\sum\limits_{h=1}^L W_h \bar{x}_h\mbox{,}
\end{equation}
where $L$ is the number of strata. The estimated variance of a particular strata $h$ can be computed by
\begin{equation}
\sigma^2_{hx}=\frac{\sum\limits_{i=1}^{n_h}(x_{h,i}-\bar{x}_h)^2}{n_h-1}\mbox{.}
\end{equation}
The standard error estimate for the entire population is then given by:
\begin{equation}
SE(\bar{x})=\sqrt{\sum\limits_{h=1}^L \left( \frac{N_h}{N} \right)^2\frac{\sigma^2_{hx}}{n_h} \left( \frac{N_h-n_h}{N_h} \right) }\mbox{.}
\end{equation}

When calculating a proportion (e.g. percentage of satisfied (3, 4, or 5 on the Likert scale) or highly
satisfied (4 or 5 on the Likert scale) users) estimated proportion for a given sample in a strata $h$, $p_{hy}$, with $y$ being the
dichotomous variable of either ``satisfied'' or ``not satisfied'' is given by:
\begin{equation}
p_{hy}=\frac{\sum\limits_{i=1}^{n_h} y_{hi}}{n_h}\mbox{.}
\end{equation}
The proportion for the entire population is then given by:
\begin{equation}
p_y=\frac{\sum\limits_{h=1}^L N_h p_{hy}}{N}\mbox{,}
\end{equation}
with standard error:
\begin{equation}
SE(p_y)=\sqrt{\sum\limits_{h=1}^L\left(\frac{N_h}{N}\right)^2\frac{p_{hy}\left(1-p_{hy}\right)}{n_h-1}\left(\frac{N_h-n_h}{N_h}\right)}
\end{equation}
As always, 95\% confidence intervals can be computed for a variable $x$ by $x \pm 1.96 SE(x)$.

%----------------------------------------------------------------------------------------
%	BIBLIOGRAPHY
%----------------------------------------------------------------------------------------

\bibliographystyle{unsrt}

\bibliography{SurveyMethods}

%----------------------------------------------------------------------------------------

\end{document}